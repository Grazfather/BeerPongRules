%--------------------------------------------------------
% BEER PONG RULES: SETUP SECTION
%--------------------------------------------------------
\section{Setup}\label{sec:SETUP}
	\subsection{Cup Formation}\label{ssec:CupFormation}
		\begin{enumerate}[label=(\roman*), ref=\roman*]
            \item \label{sssec:CF,solocups} Each player's rack is to be made up of 10 red \US{18}{\USoz} solo cups.
                See \hyperref[fig:solocup]{Figure \ref*{fig:solocup}} for the regulation cup specifications. 
            \item \label{sssec:CF,triangle} The starting rack is a tight equilateral triangle with the base towards the player's table edge and the tip towards the opponent. 
            \item \label{sssec:CF,position} The rack's back edge is to be at least two finger widths from the edge but no more than 4 and centred from side to side. 
            \item \label{sssec:CF,kissing} Solo Cup rims must be ``kissing". There should be minimal spaces in between the cups but not so close as to lean, tilt, or overlap onto their neighbouring cups. 
        \end{enumerate}
        \begin{figure}[H]% Shows a single rack setup
            \centering
            \def\svgwidth{0.4\columnwidth}
            \input{Figures/startrack.pdf_tex}
            \caption{Diagram of a full 10 rack of cups, centred on the table. The arrow points towards the opponent's rack. The hashed middle cup is the bitch cup (see \hyperref[ssec:BitchCup]{Section \ref*{ssec:BitchCup}}).}
            \label{fig:therack}
        \end{figure}
	\subsection{Cup Content}\label{ssec:CupContent}
        \begin{enumerate}[label=(\roman*), ref=\roman*]
            \item \label{sssec:CC,filling} Cups are to be filled to a minimum that stops the moving and sliding when a ball is sunk into the cup. 
            \item \label{sssec:CC,w_vs_l} The physical content of the cup is either water or liquor.
                \begin{enumerate}[label=(\alph*), leftmargin=2cm]
                    \item Water Cups: Water is put into the cups following \hyperref[sssec:CC,filling]{Section \ref*{sec:SETUP}.\ref*{ssec:CupContent}.\ref*{sssec:CC,filling}} regulations.
                    \item Drink Cups: Drinks are put into the solo cups following \hyperref[sssec:CC,filling]{Section \ref*{ssec:CupContent}.\ref*{sssec:CC,filling}} regulations and budget costs...
                \end{enumerate} 
            \item \label{sssec:CC,rinse} A cup of water will be provided to the players to rinse the balls off when playing with beer. This is for sanitary reasons. 
        \end{enumerate}        
    \subsection{The Balls}\label{ssec:Balls}
        \begin{enumerate}[label=(\roman*), ref=\roman*]
            \item \label{sssec:Balls,num} The game is played with 2 \US{1.57}{\inch} or \US{40}{\milli\metre} diameter, Ping Pong Balls (see \hyperref[fig:solocup]{Figure \ref*{fig:solocup}}).
            \item \label{sssec:Balls,dents} The balls must have no dents. 
            \item \label{sssec:Balls,bounce} The balls must be checked to have the same elasticity (bounce). 
            \item \label{sssec:Balls,texture} The texture of the balls must be consistent between the two; to the player's standards. This is for consistency of throwing. 
        \end{enumerate}    
	\subsection{The Cup}\label{ssec:Cup}
        \begin{enumerate}[label=(\roman*), ref=\roman*]
            \item \label{sssec:Cup,dim} The solo cups are to be red, \US{18}{\USoz} and have the dimensions as seen in \hyperref[fig:solocup]{Figure \ref*{fig:solocup}}. 
            \item \label{sssec:Cup,broken} Cups should not be used if broken or cracked in any way. Change this cup out. 
            \item \label{ssec:Cup,rinsing} Cups should be rinsed before a game in both water and drink versions before play. Cups can get quite gross. 
        \end{enumerate}
        \begin{figure}[H]
            \centering
            \def\svgwidth{\columnwidth}
            \input{Figures/CupDimensions.pdf_tex}
            \caption{Dimensions of a regulation \US{18}{\ounce} Solo Cup and Regulation ping pong ball.}
            \label{fig:solocup}
        \end{figure}
	\subsection{Teams}\label{ssec:Teams}
		\begin{enumerate}[label=(\roman*), ref=\roman*]
            \item \label{sssec:teams,options} There are two options to play beer pong: single or doubles. 
                \begin{enumerate}[label=(\alph*), leftmargin=2cm]% Singles vs doubles
                    \item \textbf{Singles:} Beer pong can played as singles where for each rack there is one player.
                        This player will throw two balls each go. 
                        The two balls (first and second throw) are independent of rules such as ``on-fire" and ``bounce".
                    \item \textbf{Doubles:}	This is a variation still played with two ping pong balls where each member of the team now throws a single ball.
                        Again, each ball is considered independent for rules such as ``on-fire" and ``bounce" but this time it is by player and not throw order.
                \end{enumerate} 
            \item \label{sssec:teams,choosing} There are no set rules for how teams are formed. Figure it out bud. 
        \end{enumerate}
	\subsection{The Table}\label{ssec:Table}
        \begin{enumerate}[label=(\roman*), ref=\roman*]
            \item \label{sssec:Table,sides} The table is split into two sides down the midline (see \hyperref[fig:table]{Figure \ref*{fig:table}}).
			    The opponents take opposite sides of the table. 
			    From the midpoint on the table in the direction the player is considered ``their side". 
            \item \label{sssec:Table,switch} If multiple games in a row are being played by the same players, they must switch sides every game. 
            \item \label{sssec:Table,length} The table must be at least \US{5}{\feet} long. 
        \end{enumerate}
        \begin{figure}[H]
            \centering
            \def\svgwidth{\columnwidth}
            \input{Figures/table.pdf_tex}
            \caption{The table viewed from the top showing two racks, accepted throwing zones, the midpoint, and the 2 fingers from the edge that the rack has to be.}
            \label{fig:table}
        \end{figure}
